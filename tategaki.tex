\documentclass[a4j,12pt,twocolumn]{ltjtarticle}

\usepackage{luatexja}
\usepackage[no-math]{luatexja-fontspec}
\usepackage[sourcehan-ja]{luatexja-preset}
\usepackage{luatexja-ruby}
\usepackage{luatexja-otf}
\renewcommand{\thefootnote}{\arabic{footnote})}
\title{縦書きのテスト}
\author{何之 某}
\date{}

\begin{document}
\maketitle
\section{はじめに}
\footnote{http://www.fugenji.org/~thomas/texlive-guide/vertical.html}これは日本語を書くためのテストです。\\
\ruby{鵲}{かささぎ}の渡せる橋におく霜の 白きを見れば見れば夜も更にける\\
\rightline{家持}
\section{短いですが}

例はこんな感じです。

12個で1ダース、\rensuji{144}個で1グロス。

12個で1ダース、144個で1グロス。

%\twocolumn

首相会見が開かれれば、司会の\footnote{これは脚注です}山田氏絡みの質問も相次ぐだろう。首相は会見先送りについて「山田広報官のことは全く関係ない」と強調したが、やはり「山田氏隠し」の疑いはぬぐい切れない。

緊急事態宣言の「全面解除」をめぐる今後の首相会見で、山田氏は司会進行の職責を果たせるだろうか。

そう思っていたところ、3月1日朝に「山田氏入院」の速報が飛び込んできた。さらに、山田氏が辞意を伝えたことを受け、政府は同日朝の持ち回り閣議で辞任を決定した。与野党の批判を受ける中での事実上の引責とみられ、山田氏を続投させた首相の判断も厳しく問われよう。

\rensuji{12}個で\rensuji{1}ダース、\rensuji{144}個で\rensuji{1}グロス。

\pbox<y>{\CID{7608}横書きの引用符\CID{7609}}\pbox<y>{〝横書きの引用符〞}

\CID{7956}縦書きの引用符\CID{7957}〝縦書きの引用符〞

\bou{傍点}、\ltjruby{圏|点}{・|・}、\kasen{傍線}。
\ltjruby[tbaseheight=0.5]{本田|侑汰}{ほんだ|うた}\\
\ltjruby{鵲}{かささぎ}のわたせる橋にをく霜の白きをみれば夜も更けにける

\noindent
\ltjruby[tbaseheight=0.88]{本}{ほん}\
\ltjruby[tbaseheight=0.88]{dvi}{ディー ヴィー アイ}\\
\ltjruby[tbaseheight=0.5]{本}{ほん}\
\ltjruby[tbaseheight=0.5]{dvi}{ディー ヴィー アイ}\\
%\ltjruby[tbaseheight=0]{本}{ほん}\
%\ltjruby[tbaseheight=0]{dvi}{ディー ヴィー アイ}\\
\ltjruby[tbaseheight=-1]{本}{ほん}
%\ltjruby[tbaseheight=-1]{dvi}{ディー ヴィー アイ\\%これは出力されない!注意!

\end{document}
